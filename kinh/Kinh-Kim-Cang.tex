
	\title{KIM CANG BÁT-NHÃ BA-LA-MẬT KINH \\ Dao-Tần Tam-tạng pháp sư Cưu-Ma-La-Thập dịch}
	
	\vspace{1em}

	
	\section*{PHÁP HỘI NHÂN DO PHẦN ĐỆ NHẤT}
	
	Như thị ngã văn : 
	
	Nhất thời Phật tại Xá-vệ quốc, Kỳ thọ Cấp Cô Độc viên, dữ đại Tỳ-kheo chúng thiên nhị bách ngũ thập nhân câu. 
	
	Nhĩ thời, Thế Tôn thực thời, trước y trì bát, nhập Xá-vệ đại thành khất thực. 
	
	Ư kỳ thành trung thứ đệ khất dĩ, hoàn chí bản xứ. 
	
	Phạn thực ngật, thu y bát. 
	
	Tẩy túc dĩ, phu tòa nhi tọa.
	
	\section*{THIỆN HIỆN KHẢI THỈNH PHẦN ĐỆ NHỊ}
	
	Thời trưởng lão Tu-bồ-đề tại đại chúng trung, tức tùng tòa khởi, thiên đản hữu kiên, hữu tất trước địa, hiệp chưởng cung kính, nhi bạch Phật ngôn :
	
	« Hi hữu, Thế Tôn ! Như Lai thiện hộ niệm chư Bồ-tát, thiện phó chúc chư Bồ-tát. 
	
	Thế Tôn ! Thiện nam tử, thiện nữ nhân, phát A-nậu-đa-la tam-miệu tam-bồ-đề tâm, vân hà ưng trụ, vân hà hàng phục kỳ tâm ? »
	
	Phật ngôn :
	
	« Thiện tai, thiện tai ! Tu-bồ-đề ! như nhữ sở thuyết, Như Lai thiện hộ niệm chư Bồ-tát, thiện phó chúc chư Bồ-tát. 
	
	Nhữ kim đế thính, đương vị nhữ thuyết. 
	
	Thiện nam tử, thiện nữ nhân phát A-nậu-đa-la tam-miệu tam-bồ-đề tâm, ưng như thị trụ, như thị hàng phục kỳ tâm. »
	
	« Duy nhiên, Thế Tôn ! Nguyện nhạo dục văn. »
	
	\section*{ĐẠI THỪA CHÁNH TÔNG PHẦN ĐỆ TAM}
	
	Phật cáo Tu-bồ-đề :
	
	« Chư Bồ-tát ma-ha-tát, ưng như thị hàng phục kỳ tâm. 
	
	Sở hữu nhất thiết chúng sanh chi loại—nhược noãn sanh, nhược thai sanh, nhược thấp sanh, nhược hóa sanh ; 
	
	nhược hữu sắc, nhược vô sắc ; 
	
	nhược hữu tưởng, nhược vô tưởng ; 
	
	nhược phi hữu tưởng phi vô tưởng, ngã giai linh nhập Vô dư Niết-bàn nhi diệt độ chi. 
	
	Như thị diệt độ vô lượng, vô số, vô biên chúng sanh, thực vô chúng sanh đắc diệt độ giả. 
	
	Hà dĩ cố ? Tu-bồ-đề ! nhược Bồ-tát hữu ngã tướng, nhân tướng, chúng sanh tướng, thọ giả tướng, tức phi Bồ-tát. »
	
	\section*{DIỆU HẠNH VÔ TRỤ PHẦN ĐỆ TỨ}
	
	« Phục thứ, Tu-bồ-đề ! Bồ-tát ư pháp, ưng vô sở trụ, hành ư bố thí ; 
	
	Sở vị bất trụ sắc bố thí, bất trụ thanh hương vị xúc pháp bố thí. 
	
	Tu-bồ-đề ! Bồ-tát ưng như thị bố thí, bất trụ ư tướng. 
	
	Hà dĩ cố ? Nhược Bồ-tát bất trụ tướng bố thí, kỳ phước đức bất khả tư lương. 
	
	Tu-bồ-đề ! Ư ý vân hà ? Đông phương hư không khả tư lương phủ ?
	
	« Phất dã, Thế Tôn !
	
	« Tu-bồ-đề ! Nam Tây Bắc phương, tứ duy, thượng hạ hư không, khả tư lương phủ ? »
	
	« Phất dã, Thế Tôn ! »
	
	« Tu-bồ-đề ! Bồ-tát vô trụ tướng bố thí, phước đức diệc phục như thị bất khả tư lương. 
	
	Tu-bồ-đề ! Bồ-tát đản ưng như sở giáo trụ. »
	
	\section*{NHƯ LÝ THẬT KIẾN PHẦN ĐỆ NGŨ}
	
	« Tu-bồ-đề ! Ư ý vân hà ? Khả dĩ thân tướng kiến Như Lai phủ ? »
	
	« Phất dã, Thế Tôn ! Bất khả dĩ thân tướng đắc kiến Như Lai. 
	
	Hà dĩ cố ? Như Lai sở thuyết thân tướng, tức phi thân tướng. »
	
	Phật cáo Tu-bồ-đề :
	
	« Phàm sở hữu tướng, giai thị hư vọng. 
	
	Nhược kiến chư tướng phi tướng, tắc kiến Như Lai. »
	
	\section*{CHÁNH TÍN HI HỮU PHẦN ĐỆ LỤC}
	
	Tu-bồ-đề bạch Phật ngôn :
	
	« Thế Tôn ! Phả hữu chúng sanh, đắc văn như thị ngôn thuyết, chương cú, sanh thật tín phủ ? »
	
	Phật cáo Tu-bồ-đề :
	
	« Mạc tác thị thuyết. Như Lai diệt hậu, hậu ngũ bách tuế, hữu trì giới tu phước giả, ư thử chương cú, năng sanh tín tâm, dĩ thử vi thật. 
	
	Đương tri thị nhân, bất ư nhất Phật, nhị Phật, tam tứ ngũ Phật nhi chủng thiện căn, dĩ ư vô lượng thiên vạn Phật sở chủng chư thiện căn. 
	
	Văn thị chương cú, nãi chí nhất niệm sanh tịnh tín giả ; Tu-bồ-đề ! Như Lai tất tri tất kiến, thị chư chúng sanh, đắc như thị vô lượng phước đức. 
	
	Hà dĩ cố ? Thị chư chúng sanh, vô phục ngã tướng, nhân tướng, chúng sanh tướng, thọ giả tướng, vô pháp tướng, diệc vô phi pháp tướng. 
	
	Hà dĩ cố ? Thị chư chúng sanh, nhược tâm thủ tướng, tắc vi trước ngã, nhân, chúng sanh, thọ giả. 
	
	Nhược thủ pháp tướng, tức trước ngã, nhân, chúng sanh, thọ giả.
	
	Hà dĩ cố ? Nhược thủ phi pháp tướng, tức trước ngã, nhân, chúng sanh, thọ giả. 
	
	Thị cố bất ưng thủ pháp, bất ưng thủ phi pháp. 
	
	Dĩ thị nghĩa cố, Như Lai thường thuyết : 
	
	Nhữ đẳng Tỳ-kheo ! Tri ngã thuyết pháp, như phiệt dụ giả ; pháp thượng ưng xả, hà huống phi pháp. »
	
	\section*{VÔ ĐẮC VÔ THUYẾT PHẦN ĐỆ THẤT}
	
	« Tu-bồ-đề ! Ư ý vân hà ? Như Lai đắc A-nậu-đa-la tam-miệu tam-bồ-đề da ? Như Lai hữu sở thuyết pháp da ? »
	
	Tu-bồ-đề ngôn :
	
	« Như ngã giải Phật sở thuyết nghĩa, vô hữu định pháp, danh A-nậu-đa-la tam-miệu tam-bồ-đề ; diệc vô hữu định pháp Như Lai khả thuyết. 
	
	Hà dĩ cố ? Như Lai sở thuyết pháp, giai bất khả thủ, bất khả thuyết ; phi pháp, phi phi pháp. 
	
	Sở dĩ giả hà ? Nhất thiết Hiền Thánh, giai dĩ vô vi pháp, nhi hữu sai biệt.
	
	\section*{Y PHÁP XUẤT SANH PHẦN ĐỆ BÁT}
	
	« Tu-bồ-đề ! Ư ý vân hà ? Nhược nhân mãn tam thiên đại thiên thế giới thất bảo, dĩ dụng bố thí. Thị nhân sở đắc phước đức, ninh vi đa phủ ? »
	
	Tu-bồ-đề ngôn : 
	
	« Thậm đa, Thế Tôn ! Hà dĩ cố ? Thị phước đức, tức phi phước đức tánh. Thị cố Như Lai thuyết phước đức đa. »
	
	« Nhược phục hữu nhân, ư thử kinh trung, thọ trì nãi chí tứ cú kệ đẳng, vị tha nhân thuyết, kỳ phước thắng bỉ. 
	
	Hà dĩ cố ? Tu-bồ-đề ! Nhất thiết chư Phật, cập chư Phật A-nậu-đa-la tam-miệu tam-bồ-đề pháp, giai tùng thử kinh xuất. 
	
	Tu-bồ-đề ! Sở vị Phật pháp giả, tức phi Phật pháp.
	
	\section*{NHẤT TƯỚNG VÔ TƯỚNG PHẦN ĐỆ CỬU}
	
	« Tu-bồ-đề ! Ư ý vân hà ? Tu-đà-hoàn năng tác thị niệm, ngã đắc Tu-đà-hoàn quả phủ ? »
	
	Tu-bồ-đề ngôn :
	
	« Phất dã, Thế Tôn ! Hà dĩ cố ? Tu-đà-hoàn danh vi Nhập lưu, nhi vô sở nhập ; bất nhập sắc, thanh, hương, vị, xúc, pháp, thị danh Tu-đà-hoàn. »
	
	« Tu-bồ-đề ! Ư ý vân hà ? Tư-đà-hàm năng tác thị niệm, ngã đắc Tư-đà-hàm quả phủ ? »
	
	Tu-bồ-đề ngôn :
	
	« Phất dã, Thế Tôn ! Hà dĩ cố ? Tư-đà-hàm danh Nhất vãng lai, nhi thật vô vãng lai, thị danh Tư-đà-hàm.
	
	« Tu-bồ-đề ! Ư ý vân hà ? A-na-hàm năng tác thị niệm, ngã đắc A-na-hàm quả phủ ? »
	
	Tu-bồ-đề ngôn :
	
	« Phất dã, Thế Tôn ! Hà dĩ cố ? A-na-hàm danh vi Bất lai, nhi thật vô bất lai, thị cố danh A-na-hàm. »
	
	« Tu-bồ-đề ! Ư ý vân hà ? A-la-hán năng tác thị niệm, ngã đắc A-la-hán đạo phủ ? »
	
	Tu-bồ-đề ngôn :
	
	« Phất dã, Thế Tôn ! Hà dĩ cố ? Thật vô hữu pháp danh A-la-hán. 
	
	Thế Tôn ! Nhược A-la-hán tác thị niệm : ngã đắc A-la-hán đạo, tức vi trước ngã, nhân, chúng sanh, thọ giả. 
	
	Thế Tôn ! Phật thuyết ngã đắc Vô tránh tam-muội, nhân trung tối vi đệ nhất, thị đệ nhất ly dục A-la-hán. 
	
	Thế Tôn ! Ngã bất tác thị niệm : ngã thị ly dục A-la-hán. 
	
	Thế Tôn ! Ngã nhược tác thị niệm : ngã đắc A-la-hán đạo, Thế Tôn tắc bất thuyết Tu-bồ-đề thị nhạo A-lan-na hạnh giả ; dĩ Tu-bồ-đề thật vô sở hành, nhi danh Tu-bồ-đề, thị nhạo A-lan-na hạnh. »
	
	\section*{TRANG NGHIÊM TỊNH ĐỘ PHẦN ĐỆ THẬP}
	
	Phật cáo Tu-bồ-đề :
	
	« Ư ý vân hà ? Như Lai tích tại Nhiên Đăng Phật sở, ư pháp hữu sở đắc phủ ? »
	
	« Phất dã, Thế Tôn ! Như Lai tại Nhiên Đăng Phật sở, ư pháp thật vô sở đắc. »
	
	« Tu-bồ-đề ! Ư ý vân hà ? Bồ-tát trang nghiêm Phật độ phủ ? »
	
	« Phất dã, Thế Tôn ! Hà dĩ cố ? Trang nghiêm Phật độ giả, tức phi trang nghiêm, thị danh trang nghiêm. »
	
	« Thị cố, Tu-bồ-đề ! Chư Bồ-tát ma-ha-tát, ưng như thị sanh thanh tịnh tâm, bất ưng trụ sắc sanh tâm, bất ưng trụ thanh, hương, vị, xúc, pháp sanh tâm, ưng vô sở trụ, nhi sanh kỳ tâm. 
	
	Tu-bồ-đề ! Thí như hữu nhân, thân như Tu-di sơn vương, ư ý vân hà ? Thị thân vi đại phủ ? »
	
	Tu-bồ-đề ngôn :
	
	« Thậm đại, Thế Tôn ! Hà dĩ cố ? Phật thuyết phi thân, thị danh đại thân. »
	
	\section*{VÔ VI PHƯỚC THẮNG PHẦN ĐỆ THẬP NHẤT}
	
	« Tu-bồ-đề ! Như Hằng hà trung sở hữu sa số, như thị sa đẳng Hằng hà. Ư ý vân hà ? Thị chư Hằng hà sa, ninh vi đa phủ ? »
	
	Tu-bồ-đề ngôn :
	
	« Thậm đa, Thế Tôn ! Đản chư Hằng hà, thượng đa vô số, hà huống kỳ sa. »
	
	« Tu-bồ-đề ! Ngã kim thật ngôn cáo nhữ, nhược hữu thiện nam tử, thiện nữ nhân, dĩ thất bảo mãn nhĩ sở Hằng hà sa số tam thiên đại thiên thế giới, dĩ dụng bố thí, đắc phước đa phủ ? »
	
	Tu-bồ-đề ngôn :
	
	« Thậm đa, Thế Tôn ! »
	
	Phật cáo Tu-bồ-đề :
	
	« Nhược thiện nam tử, thiện nữ nhân, ư thử kinh trung, nãi chí thọ trì tứ cú kệ đẳng, vị tha nhân thuyết, nhi thử phước đức, thắng tiền phước đức. »
	
	\section*{TÔN TRỌNG CHÁNH GIÁO PHẦN ĐỆ THẬP NHỊ}
	
	« Phục thứ, Tu-bồ-đề ! Tùy thuyết thị kinh, nãi chí tứ cú kệ đẳng, đương tri thử xứ, nhất thiết thế gian thiên, nhân, a-tu-la, giai ưng cúng dường, như Phật tháp miếu. 
	
	Hà huống hữu nhân, tận năng thọ trì, độc tụng. 
	
	Tu-bồ-đề ! Đương tri thị nhân, thành tựu tối thượng đệ nhất hi hữu chi pháp. 
	
	Nhược thị kinh điển sở tại chi xứ, tức vi hữu Phật, nhược tôn trọng đệ tử. »
	
	\section*{NHƯ PHÁP THỌ TRÌ PHẦN ĐỆ THẬP TAM}
	
	Nhĩ thời, Tu-bồ-đề bạch Phật ngôn :
	
	« Thế Tôn ! Đương hà danh thử kinh? Ngã đẳng vân hà phụng trì ? »
	
	Phật cáo Tu-bồ-đề :
	
	« Thị kinh danh vi Kim Cang Bát-nhã Ba-la-mật, dĩ thị danh tự, nhữ đương phụng trì. 
	
	Sở dĩ giả hà ? Tu-bồ-đề ! Phật thuyết Bát-nhã ba-la-mật, tức phi Bát-nhã ba-la-mật, thị danh Bát-nhã ba-la-mật. 
	
	Tu-bồ-đề ! Ư ý vân hà ? Như Lai hữu sở thuyết pháp phủ ?
	
	Tu-bồ-đề bạch Phật ngôn :
	
	« Thế Tôn ! Như Lai vô sở thuyết. »
	
	« Tu-bồ-đề ! Ư ý vân hà ? Tam thiên đại thiên thế giới sở hữu vi trần, thị vi đa phủ ? »
	
	Tu-bồ-đề ngôn :
	
	« Thậm đa, Thế Tôn ! »
	
	« Tu-bồ-đề ! Chư vi trần, Như Lai thuyết phi vi trần, thị danh vi trần. 
	
	Như Lai thuyết thế giới phi thế giới, thị danh thế giới. 
	
	Tu-bồ-đề ! Ư ý vân hà ? Khả dĩ tam thập nhị tướng kiến Như Lai phủ ? »
	
	« Phất dã, Thế Tôn ! Bất khả dĩ tam thập nhị tướng đắc kiến Như Lai. 
	
	Hà dĩ cố ? Như Lai thuyết tam thập nhị tướng, tức thị phi tướng, thị danh tam thập nhị tướng. »
	
	« Tu-bồ-đề ! Nhược hữu thiện nam tử, thiện nữ nhân, dĩ Hằng hà sa đẳng thân mạng bố thí, nhược phục hữu nhân, ư thử kinh trung, nãi chí thọ trì tứ cú kệ đẳng, vị tha nhân thuyết, kỳ phước thậm đa ! »
	
	\section*{PHẦN 14. LY TƯỚNG TỊCH DIỆT PHẦN ĐỆ THẬP TỨ}
	
	Nhĩ thời, Tu-bồ-đề văn thuyết thị kinh, thâm giải nghĩa thú, thế lệ bi khấp, nhi bạch Phật ngôn :
	
	« Hi hữu, Thế Tôn ! Phật thuyết như thị thậm thâm kinh điển, ngã tùng tích lai sở đắc tuệ nhãn, vị tằng đắc văn như thị chi kinh.
	
	Thế Tôn ! Nhược phục hữu nhân đắc văn thị kinh, tín tâm thanh tịnh, tắc sanh Thật tướng ; đương tri thị nhân thành tựu đệ nhất hi hữu công đức. 
	
	Thế Tôn ! Thị thật tướng giả, tức thị phi tướng, thị cố Như Lai thuyết danh thật tướng.
	
	Thế Tôn ! Ngã kim đắc văn như thị kinh điển, tín giải thọ trì, bất túc vi nan. Nhược đương lai thế, hậu ngũ bách tuế, kỳ hữu chúng sanh, đắc văn thị kinh, tín giải thọ trì, thị nhân tắc vi đệ nhất hi hữu. 
	
	Hà dĩ cố ? Thử nhân vô ngã tướng, vô nhân tướng, vô chúng sanh tướng, vô thọ giả tướng. 
	
	Sở dĩ giả hà ? Ngã tướng, tức thị phi tướng ; nhân tướng, chúng sanh tướng, thọ giả tướng tức, thị phi tướng. 
	
	Hà dĩ cố ? Ly nhất thiết chư tướng, tức danh chư Phật.
	
	Phật cáo Tu-bồ-đề :
	
	« Như thị, như thị ! Nhược phục hữu nhân, đắc văn thị kinh, bất kinh, bất bố, bất úy, đương tri thị nhân, thậm vi hi hữu ! 
	
	Hà dĩ cố ? Tu-bồ-đề ! Như Lai thuyết đệ nhất ba-la-mật, tức phi đệ nhất ba-la-mật, thị danh đệ nhất ba-la-mật. 
	
	Tu-bồ-đề ! Nhẫn nhục ba-la-mật, Như Lai thuyết phi nhẫn nhục ba-la-mật, thị danh nhẫn nhục ba-la-mật. 
	
	Hà dĩ cố ? Tu-bồ-đề ! Như ngã tích vi Ca-lợi vương cát tiệt thân thể, ngã ư nhĩ thời, vô ngã tướng, vô nhân tướng, vô chúng sanh tướng, vô thọ giả tướng. 
	
	Hà dĩ cố ? Ngã ư vãng tích tiết tiết chi giải thời, nhược hữu ngã tướng, nhân tướng, chúng sanh tướng, thọ giả tướng, ưng sanh sân hận.
	
	Tu-bồ-đề ! Hựu niệm quá khứ, ư ngũ bách thế, tác nhẫn nhục tiên nhân, ư nhĩ sở thế, vô ngã tướng, vô nhân tướng, vô chúng sanh tướng, vô thọ giả tướng.
	
	Thị cố, Tu-bồ-đề ! Bồ-tát ưng ly nhất thiết tướng, phát A-nậu-đa-la tam-miệu tam-bồ-đề tâm, bất ưng trụ sắc sanh tâm, bất ưng trụ thanh hương vị xúc pháp sanh tâm, ưng sanh vô sở trụ tâm. 
	
	Nhược tâm hữu trụ, tắc vi phi trụ. 
	
	Thị cố Phật thuyết Bồ-tát tâm, bất ưng trụ sắc bố thí. 
	
	Tu-bồ-đề ! Bồ-tát vị lợi ích nhất thiết chúng sanh cố, ưng như thị bố thí. 
	
	Như Lai thuyết : nhất thiết chư tướng, tức thị phi tướng. 
	
	Hựu thuyết : nhất thiết chúng sanh, tức phi chúng sanh.
	
	Tu-bồ-đề ! Như Lai thị chân ngữ giả, thật ngữ giả, như ngữ giả, bất cuống ngữ giả, bất dị ngữ giả. 
	
	Tu-bồ-đề ! Như Lai sở đắc pháp, thử pháp vô thật vô hư.
	
	Tu-bồ-đề ! Nhược Bồ-tát tâm trụ ư pháp, nhi hành bố thí, như nhân nhập ám, tắc vô sở kiến. 
	
	Nhược Bồ-tát tâm bất trụ pháp, nhi hành bố thí, như nhân hữu mục, nhật quang minh chiếu, kiến chủng chủng sắc.
	
	Tu-bồ-đề ! Đương lai chi thế, nhược hữu thiện nam tử, thiện nữ nhân, năng ư thử kinh thọ trì độc tụng, tức vi Như Lai, dĩ Phật trí tuệ, tất tri thị nhân, tất kiến thị nhân, giai đắc thành tựu vô lượng vô biên công đức. »
	
	\section*{TRÌ KINH CÔNG ĐỨC PHẦN ĐỆ THẬP NGŨ}
	
	« Tu-bồ-đề ! Nhược hữu thiện nam tử thiện nữ nhân, sơ nhật phần, dĩ Hằng hà sa đẳng thân bố thí ; trung nhật phần phục dĩ Hằng hà sa đẳng thân bố thí ; hậu nhật phần diệc dĩ Hằng hà sa đẳng thân bố thí ; như thị vô lượng bách thiên vạn ức kiếp,  dĩ thân bố thí.
	
	Nhược phục hữu nhân, văn thử kinh điển, tín tâm bất nghịch, kỳ phước thắng bỉ ; hà huống thư tả, thọ trì độc tụng, vị nhân giải thuyết.
	
	Tu-bồ-đề ! Dĩ yếu ngôn chi, thị kinh hữu bất khả tư nghị, bất khả xứng lượng, vô biên công đức, Như Lai vị phát Đại thừa giả thuyết, vị phát Tối thượng thừa giả thuyết. 
	
	Nhược hữu nhân năng thọ trì độc tụng, quảng vị nhân thuyết, Như Lai tất tri thị nhân, tất kiến thị nhân, giai đắc thành tựu bất khả lượng, bất khả xứng, vô hữu biên, bất khả tư nghị công đức. 
	
	Như thị nhân đẳng, tắc vi hà đảm Như Lai A-nậu-đa-la tam-miệu tam-bồ-đề. 
	
	Hà dĩ cố ? Tu-bồ-đề ! Nhược nhạo tiểu pháp giả, trước ngã kiến, nhân kiến, chúng sanh kiến, thọ giả kiến, tức ư thử kinh bất năng thính thọ độc tụng, vị nhân giải thuyết.
	
	Tu-bồ-đề ! Tại tại xứ xứ, nhược hữu thử kinh, nhất thiết thế gian thiên, nhân, a-tu-la, sở ưng cúng dường. Đương tri thử xứ, tắc vi thị tháp, giai ưng cung kính, tác lễ vi nhiễu, dĩ chư hoa hương nhi tán kỳ xứ. »
	
	\section*{NĂNG TỊNH NGHIỆP CHƯỚNG PHẦN ĐỆ THẬP LỤC}
	
	« Phục thứ, Tu-bồ-đề ! Nhược thiện nam tử, thiện nữ nhân, thọ trì, độc tụng thử kinh, nhược vi nhân khinh tiện, thị nhân tiên thế tội nghiệp, ưng đọa ác đạo.
	
	Dĩ kim thế nhân khinh tiện cố, tiên thế tội nghiệp, tắc vi tiêu diệt, đương đắc A-nậu-đa-la tam-miệu tam-bồ-đề.
	
	Tu-bồ-đề ! Ngã niệm quá khứ vô lượng a-tăng-kỳ kiếp, ư Nhiên Đăng Phật tiền, đắc trị bát bách tứ thiên vạn ức na-do-tha chư Phật, tất giai cúng dường thừa sự, vô không quá giả. 
	
	Nhược phục hữu nhân, ư hậu mạt thế, năng thọ trì độc tụng thử kinh, sở đắc công đức, ư ngã sở cúng dường chư Phật công đức, bách phần bất cập nhất, thiên vạn ức phần, nãi chí toán số thí dụ sở bất năng cập.
	
	Tu-bồ-đề ! Nhược thiện nam tử, thiện nữ nhân, ư hậu mạt thế, hữu thọ trì độc tụng thử kinh, sở đắc công đức, ngã nhược cụ thuyết giả, hoặc hữu nhân văn, tâm tắc cuồng loạn, hồ nghi bất tín. 
	
	Tu-bồ-đề ! Đương tri thị kinh nghĩa bất khả tư nghị, quả báo diệc bất khả tư nghị. »
	
	\section*{CỨU CÁNH VÔ NGÃ PHẦN ĐỆ THẬP THẤT}
	
	Nhĩ thời, Tu-bồ-đề bạch Phật ngôn :
	
	« Thế Tôn ! Thiện nam tử, thiện nữ nhân, phát A-nậu-đa-la tam-miệu tam-bồ-đề tâm, vân hà ưng trụ ? Vân hà hàng phục kỳ tâm ?
	
	Phật cáo Tu-bồ-đề :
	
	« Thiện nam tử, thiện nữ nhân, phát A-nậu-đa-la tam-miệu tam-bồ-đề tâm giả, đương sanh như thị tâm : ngã ưng diệt độ nhất thiết chúng sanh ; diệt độ nhất thiết chúng sanh dĩ, nhi vô hữu nhất chúng sanh thật diệt độ giả. 
	
	Hà dĩ cố ? Tu-bồ-đề ! Nhược Bồ-tát hữu ngã tướng, nhân tướng, chúng sanh tướng, thọ giả tướng, tức phi Bồ-tát. 
	
	Sở dĩ giả hà ? Tu-bồ-đề ! Thật vô hữu pháp, phát A-nậu-đa-la tam-miệu tam-bồ-đề tâm giả.
	
	Tu-bồ-đề ! Ư ý vân hà ? Như Lai ư Nhiên Đăng Phật sở, hữu pháp đắc A-nậu-đa-la tam-miệu tam-bồ-đề phủ ? »
	
	« Phất dã, Thế Tôn ! Như ngã giải Phật sở thuyết nghĩa, Phật ư Nhiên Đăng Phật sở, vô hữu pháp đắc A-nậu-đa-la tam-miệu tam-bồ-đề. »
	
	Phật ngôn :
	
	« Như thị, như thị ! Tu-bồ-đề ! Thật vô hữu pháp, Như Lai đắc A-nậu-đa-la tam-miệu tam-bồ-đề. 
	
	Tu-bồ-đề ! Nhược hữu pháp Như Lai đắc A-nậu-đa-la tam-miệu tam-bồ-đề giả, Nhiên Đăng Phật tắc bất dữ ngã thọ ký : Nhữ ư lai thế đương đắc tác Phật, hiệu Thích-ca Mâu-ni. 
	
	Dĩ thật vô hữu pháp, đắc A-nậu-đa-la tam-miệu tam-bồ-đề, thị cố Nhiên Đăng Phật dữ ngã thọ ký, tác thị ngôn : Nhữ ư lai thế, đương đắc tác Phật, hiệu Thích-ca Mâu-ni.  
	
	Hà dĩ cố ? Như Lai giả, tức chư pháp như nghĩa. 
	
	Nhược hữu nhân ngôn : Như Lai đắc A-nậu-đa-la tam-miệu tam-bồ-đề. 
	
	Tu-bồ-đề ! Thật vô hữu pháp, Phật đắc A-nậu-đa-la tam-miệu tam-bồ-đề. 
	
	Tu-bồ-đề ! Như Lai sở đắc A-nậu-đa-la tam-miệu tam-bồ-đề, ư thị trung, vô thật vô hư.
	
	Thị cố Như Lai thuyết : Nhất thiết pháp giai thị Phật pháp.
	
	Tu-bồ-đề ! Sở ngôn nhất thiết pháp giả, tức phi nhất thiết pháp, thị cố danh nhất thiết pháp. 
	
	Tu-bồ-đề ! Thí như nhân thân trường đại. »
	
	Tu-bồ-đề ngôn : 
	
	« Thế Tôn ! Như Lai thuyết nhân thân trường đại, tắc vi phi đại thân, thị danh đại thân. »
	
	« Tu-bồ-đề ! Bồ-tát diệc như thị. 
	
	Nhược tác thị ngôn : Ngã đương diệt độ vô lượng chúng sanh. Tức bất danh Bồ-tát.
	
	Hà dĩ cố ? Tu-bồ-đề ! Thật vô hữu pháp, danh vi Bồ-tát. 
	
	Thị cố Phật thuyết : Nhất thiết pháp, vô ngã, vô nhân, vô chúng sanh, vô thọ giả. 
	
	Tu-bồ-đề ! Nhược Bồ-tát tác thị ngôn : Ngã đương trang nghiêm Phật độ. Thị bất danh Bồ-tát. 
	
	Hà dĩ cố ? Như Lai thuyết trang nghiêm Phật độ giả, tức phi trang nghiêm, thị danh trang nghiêm. 
	
	Tu-bồ-đề ! Nhược Bồ-tát thông đạt vô ngã pháp giả, Như Lai thuyết danh chân thị Bồ-tát. »
	
	\section*{NHẤT THỂ ĐỒNG QUÁN PHẦN ĐỆ THẬP BÁT}
	
	« Tu-bồ-đề ! Ư ý vân hà ? Như Lai hữu nhục nhãn phủ ? »
	
	« Như thị, Thế Tôn ! Như Lai hữu nhục nhãn. »
	
	« Tu-bồ-đề ! Ư ý vân hà ? Như Lai hữu thiên nhãn phủ ? » 
	
	« Như thị, Thế Tôn ! Như Lai hữu thiên nhãn. »
	
	« Tu-bồ-đề ! Ư ý vân hà ? Như Lai hữu tuệ nhãn phủ ? »
	
	« Như thị, Thế Tôn ! Như Lai hữu tuệ nhãn. »
	
	« Tu-bồ-đề ! Ư ý vân hà ? Như Lai hữu pháp nhãn phủ ? »
	
	« Như thị, Thế Tôn ! Như Lai hữu pháp nhãn. »
	
	« Tu-bồ-đề ! Ư ý vân hà ? Như Lai hữu Phật nhãn phủ ? »
	
	« Như thị, Thế Tôn ! Như Lai hữu Phật nhãn. »
	
	
	« Tu-bồ-đề ! Ư ý vân hà ? Như Hằng hà trung sở hữu sa, Phật thuyết thị sa phủ ? »
	
	« Như thị, Thế Tôn ! Như Lai thuyết thị sa. »
	
	« Tu-bồ-đề ! Ư ý vân hà ? Như nhất Hằng hà trung sở hữu sa, hữu như thị sa đẳng Hằng hà, thị chư Hằng hà sở hữu sa số, Phật thế giới như thị, ninh vi đa phủ ? »
	
	« Thậm đa, Thế Tôn ! »
	
	Phật cáo Tu-bồ-đề :
	
	« Nhĩ sở quốc độ trung, sở hữu chúng sanh, nhược can chủng tâm, Như Lai tất tri. 
	
	Hà dĩ cố ? Như Lai thuyết : chư tâm giai vi phi tâm, thị danh vi tâm. 
	
	Sở dĩ giả hà ? Tu-bồ-đề ! Quá khứ tâm bất khả đắc, hiện tại tâm bất khả đắc, vị lai tâm bất khả đắc. »
	
	\section*{PHÁP GIỚI THÔNG HÓA PHẦN ĐỆ THẬP CỬU}
	
	« Tu-bồ-đề ! Ư ý vân hà ? Nhược hữu nhân mãn tam thiên đại thiên thế giới thất bảo, dĩ dụng bố thí, thị nhân dĩ thị nhân duyên, đắc phước đa phủ ? »
	
	« Như thị, Thế Tôn ! Thử nhân dĩ thị nhân duyên, đắc phước thậm đa. »
	
	« Tu-bồ-đề ! Nhược phước đức hữu thật, Như Lai bất thuyết đắc phước đức đa, dĩ phước đức vô cố, Như Lai thuyết đắc phước đức đa. »
	
	\section*{LY SẮC LY TƯỚNG PHẦN ĐỆ NHỊ THẬP}
	
	« Tu-bồ-đề ! Ư ý vân hà ? Phật khả dĩ cụ túc sắc thân kiến phủ ? »
	
	« Phất dã, Thế Tôn ! Như Lai bất ưng dĩ cụ túc sắc thân kiến. 
	
	Hà dĩ cố ? Như Lai thuyết cụ túc sắc thân, tức phi cụ túc sắc thân, thị danh cụ túc sắc thân. 
	
	« Tu-bồ-đề ! Ư ý vân hà ? Như Lai khả dĩ cụ túc chư tướng kiến phủ ? »
	
	« Phất dã, Thế Tôn ! Như Lai bất ưng dĩ cụ túc chư tướng kiến. 
	
	Hà dĩ cố ? Như Lai thuyết : chư tướng cụ túc, tức phi chư tướng cụ túc, thị danh chư tướng cụ túc. »
	
	\section*{PHI THUYẾT SỞ THUYẾT PHẦN ĐỆ NHỊ THẬP NHẤT}
	
	« Tu-bồ-đề ! Nhữ vật vị Như lai tác thị niệm : ngã đương hữu sở thuyết pháp. Mạc tác thị niệm ! 
	
	Hà dĩ cố ? Nhược nhân ngôn : Như Lai hữu sở thuyết pháp, tức vi báng Phật, bất năng giải ngã sở thuyết cố. 
	
	Tu-bồ-đề ! Thuyết pháp giả, vô pháp khả thuyết, thị danh thuyết pháp.
	
	Nhĩ thời, Tuệ Mạng Tu-bồ-đề bạch Phật ngôn :
	
	« Thế Tôn ! Phả hữu chúng sanh, ư vị lai thế, văn thuyết thị pháp, sanh tín tâm phủ ?
	
	Phật ngôn :
	
	« Tu-bồ-đề ! Bỉ phi chúng sanh, phi bất chúng sanh. 
	
	Hà dĩ cố ? Tu-bồ-đề ! Chúng sanh, chúng sanh giả, Như Lai thuyết phi chúng sanh, thị danh chúng sanh.
	
	\section*{VÔ PHÁP KHẢ ĐẮC PHẦN ĐỆ NHỊ THẬP NHỊ}
	
	Tu-bồ-đề bạch Phật ngôn :
	
	« Thế Tôn ! Phật đắc A-nậu-đa-la tam-miệu tam-bồ-đề, vi vô sở đắc da ? »
	
	Phật ngôn :
	
	« Như thị, như thị ! Tu-bồ-đề ! Ngã ư A-nậu-đa-la tam-miệu tam-bồ-đề, nãi chí vô hữu thiểu pháp khả đắc, thị danh A-nậu-đa-la tam-miệu tam-bồ-đề. »
	
	\section*{TỊNH TÂM HÀNH THIỆN PHẦN ĐỆ NHỊ THẬP TAM}
	
	« Phục thứ, Tu-bồ-đề ! Thị pháp bình đẳng, vô hữu cao hạ, thị danh A-nậu-đa-la tam-miệu tam-bồ-đề. 
	
	Dĩ vô ngã, vô nhân, vô chúng sanh, vô thọ giả, tu nhất thiết thiện pháp, tắc đắc A-nậu-đa-la tam-miệu tam-bồ-đề. 
	
	Tu-bồ-đề ! Sở ngôn thiện pháp giả, Như Lai thuyết tức phi thiện pháp, thị danh thiện pháp. »
	
	\section*{PHƯỚC TRÍ VÔ TỶ PHẦN ĐỆ NHỊ THẬP TỨ}
	
	« Tu-bồ-đề ! Nhược tam thiên đại thiên thế giới trung, sở hữu chư Tu-di sơn vương, như thị đẳng thất bảo tụ, hữu nhân trì dụng bố thí. 
	
	Nhược nhân dĩ thử Bát-nhã Ba-la-mật kinh, nãi chí tứ cú kệ đẳng, thọ trì độc tụng, vị tha nhân thuyết, ư tiền phước đức, bách phần bất cập nhất, bách thiên vạn ức phần, nãi chí toán số thí dụ sở bất năng cập. »
	
	\section*{HÓA VÔ SỞ HÓA PHẦN ĐỆ NHỊ THẬP NGŨ}
	
	« Tu-bồ-đề ! Ư ý vân hà ? Nhữ đẳng vật vị Như Lai tác thị niệm : ngã đương độ chúng sanh. 
	
	Tu-bồ-đề ! Mạc tác thị niệm. Hà dĩ cố ? Thật vô hữu chúng sanh Như Lai độ giả. 
	
	Nhược hữu chúng sanh Như Lai độ giả, Như Lai tắc hữu ngã, nhân, chúng sanh, thọ giả. 
	
	Tu-bồ-đề ! Như Lai thuyết : hữu ngã giả, tắc phi hữu ngã, nhi phàm phu chi nhân, dĩ vi hữu ngã. 
	
	Tu-bồ-đề ! Phàm phu giả, Như Lai thuyết tắc phi phàm phu, thị danh phàm phu.
	
	\section*{PHÁP THÂN PHI TƯỚNG PHẦN ĐỆ NHỊ THẬP LỤC}
	
	« Tu-bồ-đề ! Ư ý vân hà ? Khả dĩ tam thập nhị tướng quán Như Lai phủ ? »
	
	Tu-bồ-đề ngôn :
	
	« Như thị, như thị ! Dĩ tam thập nhị tướng quán Như Lai. »
	
	Phật ngôn :
	
	« Tu-bồ-đề ! Nhược dĩ tam thập nhị tướng quán Như Lai giả, Chuyển Luân Thánh vương tắc thị Như Lai. »
	
	Tu-bồ-đề bạch Phật ngôn :
	
	« Thế Tôn ! Như ngã giải Phật sở thuyết nghĩa, bất ưng dĩ tam thập nhị tướng quán Như Lai. »
	
	Nhĩ thời, Thế Tôn nhi thuyết kệ ngôn :
	
	Nhược dĩ sắc kiến ngã,Dĩ âm thanh cầu ngã,Thị nhân hành tà đạo,Bất năng kiến Như Lai.
	
	\section*{ ĐOẠN VÔ DIỆT PHẦN ĐỆ NHỊ THẬP THẤT}
	
	« Tu-bồ-đề ! Nhữ nhược tác thị niệm : Như Lai bất dĩ cụ túc tướng cố, đắc A-nậu-đa-la tam-miệu tam-bồ-đề.
	
	Tu-bồ-đề ! Mạc tác thị niệm : Như Lai bất dĩ cụ túc tướng cố, đắc A-nậu-đa-la tam-miệu tam-bồ-đề. 
	
	Tu-bồ-đề ! Nhữ nhược tác thị niệm : phát A-nậu-đa-la tam-miệu tam-bồ-đề tâm giả, thuyết chư pháp đoạn diệt. 
	
	Mạc tác thị niệm ! Hà dĩ cố ? Phát A-nậu-đa-la tam-miệu tam-bồ-đề tâm giả, ư pháp bất thuyết đoạn diệt tướng. »
	
	\section*{BẤT THỌ BẤT THAM PHẦN ĐỆ NHỊ THẬP BÁT}
	
	« Tu-bồ-đề ! Nhược Bồ-tát dĩ mãn Hằng hà sa đẳng thế giới thất bảo, trì dụng bố thí.
	
	Nhược phục hữu nhân, tri nhất thiết pháp vô ngã, đắc thành ư nhẫn. 
	
	Thử Bồ-tát thắng tiền Bồ-tát sở đắc công đức. 
	
	Hà dĩ cố ? Tu-bồ-đề ! Dĩ chư Bồ-tát bất thọ phước đức cố. »
	
	Tu-bồ-đề bạch Phật ngôn :
	
	« Thế Tôn ! Vân hà Bồ-tát, bất thọ phước đức ? »
	
	« Tu-bồ-đề ! Bồ-tát sở tác phước đức, bất ưng tham trước, thị cố thuyết : bất thọ phước đức. »
	
	\section*{UY NGHI TỊCH TĨNH PHẦN ĐỆ NHỊ THẬP CỬU}
	
	« Tu-bồ-đề ! Nhược hữu nhân ngôn : Như Lai nhược lai, nhược khứ ; nhược tọa, nhược ngọa. 
	
	Thị nhân bất giải ngã sở thuyết nghĩa.
	
	Hà dĩ cố ? Như Lai giả, vô sở tùng lai, diệc vô sở khứ, cố danh Như Lai. »
	
	\section*{NHẤT HIỆP LÝ TƯỚNG PHẦN ĐỆ TAM THẬP}
	
	« Tu-bồ-đề ! Nhược thiện nam tử, thiện nữ nhân, dĩ tam thiên đại thiên thế giới toái vi vi trần ; ư ý vân hà ? Thị vi trần chúng, ninh vi đa phủ ? »
	
	Tu-bồ-đề ngôn :
	
	« Thậm đa, Thế Tôn ! Hà dĩ cố ? Nhược thị vi trần chúng thật hữu giả, Phật tắc bất thuyết thị vi trần chúng. 
	
	Sở dĩ giả hà ? Phật thuyết vi trần chúng, tắc phi vi trần chúng, thị danh vi trần chúng.
	
	Thế Tôn ! Như Lai sở thuyết tam thiên đại thiên thế giới, tắc phi thế giới, thị danh thế giới. 
	
	Hà dĩ cố ? Nhược thế giới thật hữu giả, tắc thị nhất hiệp tướng. 
	
	Như Lai thuyết nhất hiệp tướng, tắc phi nhất hiệp tướng, thị danh nhất hiệp tướng. »
	
	« Tu-bồ-đề ! Nhất hiệp tướng giả, tắc thị bất khả thuyết ; đản phàm phu chi nhân, tham trước kỳ sự. »
	
	
	\section*{TRI KIẾN BẤT SANH PHẦN ĐỆ TAM THẬP NHẤT}
	
	« Tu-bồ-đề ! Nhược nhân ngôn : Phật thuyết ngã kiến, nhân kiến, chúng sanh kiến, thọ giả kiến.
	
	Tu-bồ-đề ! Ư ý vân hà ? Thị nhân giải ngã sở thuyết nghĩa phủ ? »
	
	« Phất dã, Thế Tôn ! Thị nhân bất giải Như Lai sở thuyết nghĩa. 
	
	Hà dĩ cố ? Thế Tôn thuyết ngã kiến, nhân kiến, chúng sanh kiến, thọ giả kiến, tức phi ngã kiến, nhân kiến, chúng sanh kiến, thọ giả kiến, thị danh ngã kiến, nhân kiến, chúng sanh kiến, thọ giả kiến. »
	
	« Tu-bồ-đề ! Phát A-nậu-đa-la tam-miệu tam-bồ-đề tâm giả, ư nhất thiết pháp, ưng như thị tri, như thị kiến, như thị tín giải, bất sanh pháp tướng. 
	
	Tu-bồ-đề ! Sở ngôn pháp tướng giả, Như Lai thuyết tức phi pháp tướng, thị danh pháp tướng. »
	
	\section*{ỨNG HÓA PHI CHÂN PHẦN ĐỆ TAM THẬP NHỊ}
	
	« Tu-bồ-đề ! Nhược hữu nhân dĩ mãn vô lượng a-tăng-kỳ thế giới thất bảo, trì dụng bố thí.
	
	Nhược hữu thiện nam tử, thiện nữ nhân, phát Bồ-đề tâm giả, trì ư thử kinh, nãi chí tứ cú kệ đẳng, thọ trì độc tụng, vị nhân diễn thuyết, kỳ phước thắng bỉ. 
	
	Vân hà vị nhân diễn thuyết ? Bất thủ ư tướng, như như bất động. Hà dĩ cố ?
	
	Nhất thiết hữu vi pháp,Như mộng, huyễn, bào, ảnh ;Như lộ diệc như điển,Ưng tác như thị quán. »
	
	Phật thuyết thị kinh dĩ, trưởng lão Tu-bồ-đề cập chư Tỳ-kheo, Tỳ-kheo ni, Ưu-bà-tắc, Ưu-bà-di, nhất thiết thế gian thiên, nhân, a-tu-la, văn Phật sở thuyết, giai đại hoan hỉ, tín thọ phụng hành.
	
	Nam-mô Bát-nhã hội-thượng Phật Bồ-tát ma-ha-tát. 
	
	\begin{center}
		\vspace{1em}
		KIM CANG BÁT-NHÃ BA-LA-MẬT KINH CHUNG
	\end{center}